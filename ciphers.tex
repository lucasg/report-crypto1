\chapter{Standards Ciphers}


\section{One-Time Pad}

\subsection{XOR operator}

$\oplus$ : XOR operator (logic gate). Widely used in cryptography

\begin{table}[ht!]
	\centering
		\begin{tabular}{c|c|c}
			$A$ & $B$ & $A\xor B$ \\
			\hline
			0 & 0 & 0 \\
			0 & 1 & 1 \\
			1 & 0 & 1 \\
			1 & 1 & 0 \\
			\hline 
		\end{tabular}
	\caption{Table of logic for XOR}
	\label{tab:TableOfLogicForXOR}
\end{table}


\subsection{Cipher}
	$ E(k,m) = c = k \oplus m $ 
	$ D(k,c ) = m = k \oplus c$
	
	
\subsection{Vulnerabilities}

Facile a cracker une fois qu'on a un m et son c => k = m XOR c
Toutefois OTP a une perfect secrecy


\section{Data Encryption Standard (DES)}
\subsection{Feistel Network}
\subsection{DESx}
