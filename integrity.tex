\chapter{Integrity}


This chapter will present the other aspect of cryptography : tampering prevention. An adversary can be able to do more than just eavesdropping : he can actually modify ciphertext messages on-the-go (using fro example a Man in the Middle attack). Therefore we need methods to ensure the non modification of the message during transmission.

\section{Message Authentification Code (MAC)}

The most common way to enfore message integrity is to add a tag to the message, which will be verified upon reception. Moreover, the tag needs to be created using a secret key in order to prevent the attacker from fooling the verification algorithm.

$MAC = (S,V)$
S : Tag Generator
    $S: (k,m) -> t$
V : Verification Algorithm
    $V(k,m,t) -> {0,1}$
    
\subsection{Construction from PRF}

A secure MAC can be easily constructed from a PRF family. The following theorem is important since a lot of real-world MAC use it, in various environment (Internet, Banks, Defense, ..).

\begin{mytheorem}
    IF $F:K\times X \leftarrow Y$ is a secure PRF and $card(Y)$ is large, then $I_F = (F, V_F)$ is a secure MAC and : \\
    $ Adv_{MAC} \leq Adv_{PRF} + \frac{1}{|Y|} $
\end{mytheorem}


\subsection{CBC-MAC}
\subsection{HMAC}
\subsection{NMAC}
\subsection{PMAC}
\subsection{Carter-Wegman MAC}


\section{Collision Resistance}
\section{Compression functions}