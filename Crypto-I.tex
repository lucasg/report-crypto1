%%%%%%%%%%%%%%%%%%%%%%%%%%%%%%%%%%%%%%%%%%%%%%%%%%%%%%%%%%%%%%%%
%% DOCUMENT %%
%%%%%%%%%%%%%%%%%%%%%%%%%%%%%%%%%%%%%%%%%%%%%%%%%%%%%%%%%%%%%%%%

% fleqn = left align equations
\documentclass[fleqn,a4paper,12pt]{book}

%%%%%%%%%%%%%%%%%%%%%%%%%%%%%%%%%%%%%%%%%%%%%%%%%%%%%%%%%%%%%%%%
%% Packages %%
%%%%%%%%%%%%%%%%%%%%%%%%%%%%%%%%%%%%%%%%%%%%%%%%%%%%%%%%%%%%%%%%

	% Page Layout
	\usepackage[top=2.54cm, bottom=2.54cm, left=2.54cm, right=2.54cm]{geometry} % margins
	
	
	% Encodings
	\usepackage[latin1]{inputenc} 	% Source text encoding
	\usepackage[T1]{fontenc}	  	% Output glyphs encoding
	\usepackage[english]{babel}	  	% Add unicode characters
	\usepackage{textcomp} 		  	% Add antoher set of characters
	%\usepackage[cyr]{aeguill}    	% hyphenated symbols
	
    % Math
    \usepackage{amsmath}			% Math package
    \usepackage{amsfonts}			% Supplementary symbols
    \usepackage[amsmath]{ntheorem}  % Theorem layouts
    \usepackage{stmaryrd}           % special symbols for natural sets

	% Graphics
	\usepackage{graphicx}			% Graph package
	\DeclareGraphicsExtensions{.pdf,.png,.PNG,.jpg,.jpeg} 
	\graphicspath{ {./images/} }    % look into images/ for images
    
    % Captions
    \usepackage[justification=centering]{caption} % long-centered captions
    \usepackage{subcaption}			% captions for subfloat
	
	% PDF
	\usepackage{pdfpages} 			% insert pdf as images
	
	% Citations
	\usepackage[round]{natbib} 		% author-date style cites
	
	% Table and Arrays
	% \usepackage{exceltex} 		% insert excels tables in tex (macro)
	\usepackage{booktabs} 			% Pretty tables
	\usepackage{array}				% for fixed-width columns and alignments 
    
    % Source Code
    \usepackage{listings}			% Insert programming langage code
    
	% URL
    \usepackage{hyperref}			% handle url and hyperlinks
    %! This package must always be included last (it's a linker operation). !%
    
%%%%%%%%%%%%%%%%%%%%%%%%%%%%%%%%%%%%%%%%%%%%%%%%%%%%%%%%%%%%%%%%
%%  Styles
%%%%%%%%%%%%%%%%%%%%%%%%%%%%%%%%%%%%%%%%%%%%%%%%%%%%%%%%%%%%%%%%

% Code Printing configuration
\lstset{
    language=Python,
    basicstyle=\small\sffamily,
    numbers=left,
    numberstyle=\tiny,
}

% Definition and Theorem layouts
\newtheorem{mydef}{Definition}

\theoremstyle{break}                  
\newtheorem{mytheorem}{Theorem}

%%%%%%%%%%%%%%%%%%%%%%%%%%%%%%%%%%%%%%%%%%%%%%%%%%%%%%%%%%%%%%%%
%% Commands
%%%%%%%%%%%%%%%%%%%%%%%%%%%%%%%%%%%%%%%%%%%%%%%%%%%%%%%%%%%%%%%%

%HRule : make a horizontal line
\newcommand{\HRule}{\rule{0.95\textwidth}{0.5mm}}

%noi = noindent
%\newcommand{\noi}{\noindent}

%alias for XOR operator.
\newcommand{\xor}{\oplus}


%%%%%%%%%%%%%%%%%%%%%%%%%%%%%%%%%%%%%%%%%%%%%%%%%%%%%%%%%%%%%%%%
%% Title Page
%%%%%%%%%%%%%%%%%%%%%%%%%%%%%%%%%%%%%%%%%%%%%%%%%%%%%%%%%%%%%%%%
\title{Cryptography I - Notes \vfill}
\author{lucasg}


% Remove every indentation (personnal style). 
\setlength\parindent{0pt}
%%%%%%%%%%%%%%%%%%%%%%%%%%%%%%%%%%%%%%%%%%%%%%%%%%%%%%%%%%%%%%%%
%% BODY
%%%%%%%%%%%%%%%%%%%%%%%%%%%%%%%%%%%%%%%%%%%%%%%%%%%%%%%%%%%%%%%%
\begin{document}

\maketitle

%%%%%%%%%%%%%%
%% Abstract
\vspace*{6cm}
\hfill
\begin{minipage}[c]{0.6\linewidth}
    \begin{flushright}
    These notes are based upon the free Coursera Course called "Cryptography I", teached by Dan Boneh from Standford, as well as information gathered from Wikipedia. It's  a personal project from an amateur in the domain : it is not exempt from errors and imprecisions. This document was written using ShareLatex, a really powerful web-based editor, and shared using Github (lucasg/crypto-report1).
    \end{flushright}
\end{minipage}


%%%%%%%%%%%%%%
%% TOC
\tableofcontents

%%%%%%%%%%%%%%
%% Chapters


\chapter{Introduction}



\noi A cipher consist of en encoder and a decoder. The algorithm behind the encoder and the decoder has to be public in order to ensure the integrity and the robustness of the cipher. The ciphers are typically  open source projects, reviewed by security experts.  \\

\subsection{Definitions}

\emph{Encoder : } $E : KxM \mapsto C$ \\
\emph{Decoder : } $D : KxC \mapsto M$ \\

\emph{Equality : } $ D(k, E(k,m) ) = m $ \\

E is often randomized whereas D is always deterministic.

\begin{figure}[ht!]
	\centering
		\includegraphics[width=0.7\textwidth]{tata}
	\caption{Cipher}
	\label{fig:Cipher}
\end{figure}

\noi A common mistake is to think that an ad-hoc crypto algorithm with closed source is safer than open source standards : the big problem with closed source algorithms is when they are breached, the user does not know.\\


\noi Uses of crypto : 
\begin{itemize}
		\item Secure communication : private conversations without eavesdropping 
		\item Digital signatures : secure identification (no tampering)
		\item Anonymous communication : secure and private communication without any of the participants know the identity of the others  
		\item Anonymous computation : outsourcing computation without giving the purpose of the calculus to the contractor (e.g. Amazon w3s)\\
\end{itemize}

\subsection{Trusted authority}

\noi A mecanism to ensure confidentiality is to outsource the task to a 3-rd party which has credibility and trust, like when we give our last will to a exterior person which does not have any involvement with the family(typically a \emph{notaire}).

\noi However, the trusted authority solution - like any centralized mecanism -  creates a single point of failure, so the trusted authority might not be always a good solution. \\

\emph{Theorem} : Any computation done by trusted auth can be done without it.\\


\subsection{ Zero proof of knowledge }
Aim of crypto : prove that, under a certain threat vector, forge the signature comes to solve a NP-problem.

\subsection{History of ciphers}

\noi cryptology is an ancient matter : all sorts of encoding scheme has been invented throughout History. \\
\emph{Definition :} $E(k,m)$ is the encryption of message $m$ using key $K$. (the key always first ) \\

\subsubsection{Substitution cipher }
The substitution cipher ( also called Julius code in its weak form ) is a simple encoder, yet relatively effective : the key $K$ is a bijective map between two alphabets. For example, $A$ becomes $D$, $B$ becomes $J$, etc.\\
The encryption is simply the substitution of every letter in the message $m$ by its counterpart in the map $K$.\\
The substitution cipher is however breakable only by looking at the ciphertexts ($=$ encryption of messages). The study of the letters' frequencies in the ciphertexts can give huge informations on the map $K$. In English, the letter $e$ is the most used so, by looking which letter is the most used in the ciphertexts, we can give a retty good guess of the subsitution of $e$.\\

\subsubsection{Vigener cipher }
breakable if we know the size of the
cipher => substation cipher.\\

\subsubsection{Enigma }
: substitution rotating cipher\\



\chapter{Theory}

\section{Perfect Secrecy}
Perfect secrecy si pour tout messages m1,m2 tel que len(m1= len(m2), 
Pr[E(k,m1) = c] = Pr[E(k,m2)=c] (k uniform in K)
Or “ Given a cypher text, I cannot tell if it closer to m1 or m2”  (no cypher text only attacks, but others attacks possible)
Corollaire : |ensKeys| => |ensMsg|

A communication system has a perfect secrecy if, for all message $m1$ and $m2$ such as $len(m1)$ = $len(m2)$ :  $ Pr[E(k,m1) = c]$ = $Pr[E(k,m2)=c]$ (k chosen uniformely in K).
In plain words, this means that , given a ciphertext,  an attacker cannot tell if it's closer to m1 or m2.

\section{Pseudo Random Generator (PRG)} 
$ G : {0,1}^s -> {0,1}^n $  with  $n>>s$, with unpredictability prorperty
G is deterministic, but the seed S is random
Stream ciphers : $c = m \oplus G(k) $  (one time pad with pseudorandom generator )


\chapter{Ciphers}


\section{One-Time Pad}

\subsection{XOR operator}

$\oplus$ : XOR operator (logic gate). Widely used in cryptography

\begin{table}[ht!]
	\centering
		\begin{tabular}{c|c|c}
			$A$ & $B$ & $A\xor B$ \\
			\hline
			0 & 0 & 0 \\
			0 & 1 & 1 \\
			1 & 0 & 1 \\
			1 & 1 & 0 \\
			\hline 
		\end{tabular}
	\caption{Table of logic for XOR}
	\label{tab:TableOfLogicForXOR}
\end{table}


\subsection{Cipher}
	$ E(k,m) = c = k \oplus m $ 
	$ D(k,c ) = m = k \oplus c$
	
	
\subsection{Vulnerabilities}

Facile a cracker une fois qu'on a un m et son c => k = m XOR c
Toutefois OTP a une perfect secrecy





\chapter{Theory}

\section{Definitions}

\subsection{Symmetric/Asymmetric Ciphers}

\begin{mydef}
\begin{minipage}[t]{0.8\textwidth}
    A cipher is symmetric if the encryption $E$ and the decryption $D$ use the same key $k$.
\end{minipage}
\end{mydef}

\begin{mydef}
\begin{minipage}[t]{0.8\textwidth}
    A cipher is asymmetric if the encryption $E$ and the decryption $D$ do not the same key. The encryption key is often called "public key" and the decryption one "private key".
\end{minipage}
\end{mydef}

The symmetric-key encryption is the oldest class of cryptography process. The major flaw of symmetric ciphers is the obligation for both parties (sender and receiver) to share the secret (the key). Nowadays, it is recommended to use non-symmetric encryption (also known as public-key encryption).

%\subsection{Computationally equivalence}
%Definition : two distributions $P_1$ and $P_2$ are computationally equivalent if , %for every "efficient" polynomial statistical test A, \\

%$ | Pr[A(X) == 1  - Pr[A(X) == 1] |  $ \\
%$   X <- P_1      X <- P_2              $ \\
%X chosen uniformly dans les distributions.

%\section{Probability Remainder}

\section{ Pseudo-Random Generation }

\subsection{Statistical test for Randomness}

Pseudorandomness is the property of a function to appear random, while being completely deterministic. In cryptography, the pseudorandomness property for a bit generator is an important one : the security is often built upon it since the attacker cannot predict the output. That's why a lot of different statistical test were conceived to separate true pseudorandom generators from broken ones (bit-frequency, chi-square Test, arithmetic mean, ...).

\begin{mydef}
A statistical test is an algorithm which, given a generator, ouptut 0 or 1 based on the stream of numbers generated, 1 being random and 0 deterministic.
\end{mydef}

\paragraph{Advantage \\}
\label{sec:advantage}

Let F an oracle\footnote{an oracle is a computational black box} which we want to study and G a perfect one implementing true randomness. The advantage $Adv$ over F, using the statistical test A, is: 
\begin{mydef}
$Adv_{F} [A,G] = | Pr[A(G(k)) == 1  - Pr[A(G(r)) == 1] | \in  [0,1] $
\end{mydef}

An advantage "close to one"\footnote{it really depends on the security margin} is considered to break the pseudorandom function, because the test A can distinguish pseudo-randomness from true randomness.


\subsection{Pseudo-Random Functions   (PRF)}

A pseudo-random function is a fairly easily computable function (~polynomial) which simulate randomness while being completely deterministic.

\paragraph{Definition \\}
Let $F$ a function from $KxD$ to $R$ which maps a two set (the domain $M$ and the range $R$) using a key parameter from $K$ : the function F is a PRF if no efficient adversary with significant advantage can distinguish F from a random oracle. Pseudo-random functions are vital tools in the construction of cryptographic primitives.

\paragraph{Secure PRF \\}
\label{sec:IND-Game-PRF}

A pseudorandom function is secure if an attacker cannot solve the following game with a significant advantage :

\begin{figure}[h!]
	\centering
		\includegraphics[width=0.7\textwidth]{secure-PRF.png}
	\caption{indistinguishability game for PRF}
	\label{fig:Cipher}
\end{figure}

Let b a binary value and $F: X\times K \rightarrow Y$  a PRF from $X$ to $Y$. If $b$ is null, the oracle will chose a random key $k\in K$, otherwise it will choose a complete random function $f:X->Y$. Then the adversary A submit one or several value(s) $x_i \in X$ and get the result $y_i \in Y$, from either the pseudo-random function or the randomly chosen one (according to the value of $b$). The attacker must find which function the oracle used, and if it find it with an advantage(see \ref{sec:advantage}), the PRF is considered insecure.


\subsection{Pseudo-Random Permutation   (PRP)}


\paragraph{Definition \\}

A PRP differs from a PRF in the way that the domain $D$ and the range $R$ are the same. Since the function map the two same sets, we can deduce that $E(k,.)$ ( k is fixed) is bijective : it is a permutation then.
A major propriety of the PRP (comparatively to the PRF) is that the PRP can be inverted, thus improving the decoding speed.\footnote{Another useful property is that permutation can be chained or cascaded since they all have the same working domain.}

\paragraph{Secure PRP \\}


\begin{figure}[ht!]
	\centering
		\includegraphics[width=0.7\textwidth]{secure-PRP.png}
	\caption{indistinguishability game for PRP}
	\label{fig:Cipher}
\end{figure}

The challenge for PRP is not much different from the one on PRF : see \ref{sec:IND-Game-PRF}.

\subsection{Pseudo Random Generator     (PRG)} 


\paragraph{Definition \\}

A pseudo random generator (PRG) is a deterministic bit generator with the property of  unpredictability :
\begin{mydef}
$ G : \llbracket  0,1 \rrbracket ^s -> \llbracket  0,1 \rrbracket ^n $  with  $n>>s$. \\ $S = \llbracket  0,1 \rrbracket ^s$ is often called "the seed" and is randomly chosen. 
\end{mydef}

Stream ciphers are easily built from PRG : $c = m \oplus G(k) $  (one time pad with pseudo-random generator )


\paragraph{Secure PRG \\}
A PRG is secure if, for all "efficient" statistical test A, $Adv(A,G)$ is negligible.

On a side note, we don't know all the existing efficient statistical tests so we can't prove that PRG is secure, we can only prove we didn't found an efficient test which break it.


\section{Confidentiality}
\subsection{Perfect Secrecy}

\begin{mytheorem}[Shannon perfect secrecy]
    $\forall m_1,m_2$ such as $len(m_1) = len(m_2)$, 
    $Pr[E(k,m_1) = c] = Pr[E(k,m_2) = c]$  \flushright (k uniform in K)
\end{mytheorem}

In layman terms, perfect secrecy means that, given two messages and the ciphertext of one of the two plaintext messages, the attacker cannot know from which message the ciphertext has been created (equal probability). A corollary is, under perfect secrecy conditions, the key space $K$ cardinality must be equal or larger than the ciphertext space $C$ cardinality, which has to be equal or larger than the message space $M$ cardinality :
\begin{mytheorem}[Shannon perfect secrecy corollary]
    $ |M| \leq |C| \leq |K| $. 
\end{mytheorem}


\subsection{Semantic Security}

The semantic security is a weaker form of Shannon's perfect secrecy (which is not usable in reality) : the distributions $P_1 = Pr[E(k,m_1) = c , k<- K]$ and $P_2$ does not have to be equal, just computationally equivalent.

\paragraph{Challenge}
The semantic security is often defined using a "challenge" : an experiment of though where and adversary has to find an important information given a certain protocol and certain rights. The challenge is the following one :

\begin{figure}[ht!]
	\centering
		\includegraphics[width=0.7\textwidth]{IND-CPA-Game.png}
	\caption{Semantic security challenge}
	\label{fig:SemanticSecurityChallenge}
\end{figure}

The challenger (the "defender") choose a fixed-key and a variable $b$ randomly from two values ( 0 and 1 to be simple). The attacker then will send the challenger $2*n$ messages (plain or encrypted) and the challenger only process half of them (given the value b) and send the result back to the attacker.\\
The attacker has to estimate the value $b$ chosen : if the attacker can guess the value of b with a significant advantage, the encryption cipher is considered insecure. Otherwise, it has semantic security against the corresponding attack channel.

\paragraph{Semantic Security against Chosen Plaintext Attack (CPA)\\}

In this configuration, the attacker can choose which plaintext message(s) to send to the challenger, and has the result of the encryption of half of them. 

\begin{mytheorem}[Semantic Security under CPA]
    $E$ is semantically secure  under chosen plaintext attack if, for all adversary A, $Adv_{CPA}[A,E] = Pr()$ is negligible.
\end{mytheorem}

\paragraph{Semantic Security against Chosen Ciphertext (Adaptive) Attack ($CCA-1$/$CCA-2$)\\}

In this configuration, the attacker can choose which ciphertext message(s) to send to the challenger, and has the result of the decryption of half of them. Under $CCA-2$ he can also makes incremental changes to the ciphertext sent given the output the previous decryption, enabling linear and differential attacks. \\
The $CCA$ assumption allow the attacker a wide range of access to information : the protocols which are secure against $CCA$ are extremely useful since they can withstand a great variety of attacks \footnote{of course, there still are side-channel attacks.}.

\section{Integrity}
\subsection{Secure MAC}


%\section{Number Theory}

\chapter{Attack Vectors}

\section{Attacks on the implementation}
\section{Fault Attacks}
\section{Linear and Differential Attacks}
\section{Exhaustive Search Attack}

\section{Quantum attacks}

\chapter{Confidentiality}

A major aim of cryptography is to provide confidentiality. Any communication can be eavesdropped : cryptography ensures that the eavesdropper does not retrieve any important information from the raw communication data.

This chapter will describe the major standard ciphers, from the one-time encryption through block cipher.

\section{One-Time Encryption}

The first family of the ciphers is the One-Time Ciphers : a cipher which use a unique key to encode a message. These ciphers are simple (yet inconvenient to use) and can reveal difficult to break (even impossible for the One-Time Pad). However, we will focus on the One-Time Pad, which is the most important one-time cipher. 

\section{One-Time Pad}

The One-Time Pad (OTP) is a symmetric cipher created for telegraph messages encryption and diplomatic communications during the beginning the twentieth century. It has been proved by Claude Shannon \footnote{Another computer science deity} to be theoretically secure. The OTP is fairly simple, yet powerful and has a lot of real-world applications (it is suggested that the OTP is used for the Washington-Moscow hotline during the Cold War).

\subsection{XOR operator}

The OTP is built from XOR operations. Since this operator has a lot of applications in cryptography, let revise it : the XOR operator is an "exclusive OR", in other words a and b can be true, but not both.


\begin{mydef}
\begin{minipage}[t]{0.8\textwidth}
\centering
    $XOR(a,b) = a \xor b = (a\&(\lnot b)) | ((\lnot a)\&b)$
\end{minipage}
\end{mydef}

\begin{table}[ht!]
	\centering
		\begin{tabular}{c|c|c}
			$A$ & $B$ & $A\xor B$ \\
			\hline
			0 & 0 & 0 \\
			0 & 1 & 1 \\
			1 & 0 & 1 \\
			1 & 1 & 0 \\
			\hline 
		\end{tabular}
	\caption{Table of logic for XOR}
	\label{tab:TableOfLogicForXOR}
\end{table}

\begin{mytheorem}
    $ a \xor a = False $
\end{mytheorem}
This corollary is really important since it's the key behind symmetric encryption using OTP.

\subsection{Cipher}

\begin{mydef}
\begin{minipage}[t]{0.8\textwidth}
	$ E(k,m) = c = k \xor m $ \\
	$ D(k,c ) = m = k \xor c$
\end{minipage}
\end{mydef}

An example of encryption :
\begin{align*}
    msg: & 0 1 1 0 1 1 1      \\
    key: & 1 0 1 1 0 1 0      \\
    cipher: & 1 1 0 1 1 0 1  \\
\end{align*}
To encrypt text-based data, one must first use a character-encoding scheme (ASCII and  EBCDIC being the most well-known) to translate characters into integer values which will be XOR'd. \\

Encryption of "HELLO" using "XMCKL" as a key, and letter's position in the alphabet as an encoding scheme :
\begin{align*}
    msg: & H E L L O        \\
    msg (binary) : & 7 4 11 11 14 \\
    key: & X M C K L   \\
    key (binay) :& 23 12 2 10 11 \\
    cipher (binay): & 30 16 13 21 25  \\
    cipher (mod 26) : & 4 16 13 21 25 \\
    cipher :& E Q N V Z  \\
\end{align*}

\begin{mytheorem}
    $ D(k, E(k,m) ) = k \xor E(k,m) = k \xor k \xor m = m $
\end{mytheorem}

The decryption takes advantage of the fact that $k \xor k$ is null. On a side note, this $\xor$ property has another consequence : $ c \xor m = k $ (the message is nullified). Given the plaintext and the ciphertext, we can recover the key (which isn't really much a problem since the key is used only once for the OTP).

\subsection{Properties}

One-time Pad's main property is to be information-theoretically unbreakable \footnote{the proof is fairly easy since $P[E(k,m) == c] = 1$ is a constant.}. In others words, the attacker , given only the ciphertext, cannot recover the plaintext message even with unlimited computing power. In the case of the ciphertext EQNVZ computed before, we can see it's the encryption of HELLO using XMCKL as key, but also the encryption of LATER using TQURI as key : given only the ciphertext, the attacker cannot recover the plaintext message or the key, since several combinations are plausible.
	
\subsection{Drawbacks}

\subsubsection{Many-time pad}
The OTP is unfortunately vulnerable to many-times encryption : if an attacker recover several cyphertexts ${c_1,c_2,...,c_n}$ encoded with the same key $k$, he is able to recover several parts of plaintext message. The attacks consists of computing $c_x\xor c_y$ operations, nullifying the encrypting key to reveal $m_x\xor m_y$. Given the character encoding scheme, it is fairly easy to reveal significant parts of text. That's why the OTP is called "One-Time" : the key must change every time there is an encryption. 

\subsubsection{Symmetrical Property}
The symmetrical nature of the One-Time pad encryption enforce the fact that the key must be shared between the emitter and the receptor. It particularly annoying in the OTP case, since the key as to change every time there is an encryption, and since the key must be long (see below).

\subsubsection{Key length}
As a consequence from the Shannon perfect secrecy theorem, the OTP key must be at least as long as the message. It is quite a bother, since half of the communication bandwidth is lost on key exchanges.

\subsubsection{Tampering}
The OTP is really weak against cyphertext tampering.

\subsection{Conclusion}

The One-Time pad is the first - and only - cipher which provides perfect secrecy, which makes him powerful but impractical since the key must be as long as the message. That's why stream ciphers (and later one block ciphers) were invented : to palliate the uneasiness of OTP's use, unfortunately at the secrecy's expense. Among actual standards of cryptography (apart from OTP), none has perfect secrecy. 

\section{ Stream Ciphers }

In order to be able to reuse the same key for several messages, two types of ciphers were created : stream ciphers and block ciphers. The stream ciphers operate on unknown length messages while block ciphers needs to known the message's size and slice it into block before encrypt it.\\
The nature of stream ciphers make them easy to use for radiocommunications \footnote{The Enigma machine is a stream cipher} : RC4 for WEP\footnote{WIFI Exchange Protocol}, A5/1 for GSM, E0 for Bluetooth, \dots Their small complexity make them also easier to port to hardware ASIC. However, there were also historically easier to break than block ciphers.\\
Stream ciphers aren't as popular as they were before compared to block ciphers, but there are not dead : there is a project called "eStream" which identify new modern and secure stream ciphers (Salsa20, Rabbit, ...). 

\section{WEP}

Let's focus on WEP since it's well known that this protocol is broken and insecure.The WEP cipher is built upon the RC4 PRG  with the following seed : 
\begin{align}
     WEP(key,message) = RC4(IV,k) \xor m 
\end{align}

The IV (initial vector) component is a packet counter : every time a packet is sent (or received), the IV counter is incremented.

\subsection{RC4 implementation}

The RC4 is implemented as a 256-bytes permutation table (whose values a initialized using the key and the key-schedule algorithm) and two 1-bytes pointers. 

\begin{figure}[hb!]
    \centering
       \includegraphics[width=\textwidth]{images/RC4.pdf}
	\caption{RC4 implementation \\ source : Wikipedia}
	\label{fig:RC4}
\end{figure}

Each cycle, i and j are incremented (modulo 256) and we perform the following operation : $k = S[ S[i] + S[j] \% 256$. Then we XOR $k$ with the current message byte to obtain the ciphertext current byte.

\subsection{Vulnerabilities}
However, according to the  this IV is coded on 24 bits, which means it overflow every 16 millions of packet cycle. 
It is a serious weakness, since we can use a two-time pad attack on packet $0$ and packet $2^24$ (which we know has the same IV and key). Moreover, at the initial handshake, the IV is fixed at 0 ! This fault completely break the WEP's security : by resetting the Access Point (AP) several times, it is fairly easy to recover the key (it is a matter of minutes).\\

\section{ Block Ciphers }

While being simpler and generally faster, the streams ciphers have security blind spots, especially against replay attacks. Block ciphers has been created to lessen this issue, while allowing multiple encoding with the same key. The block ciphers work on fixed size chunks of data - called "blocks" - and are often constructed by iterating a one-time cipher multiple times, using an initial vector sent along the ciphertext and key expansion scheme to decompress the root key.

% Insert General Schema of block cipher encryption

\subsection{Nonce and Initial Vector}

In order to be protected against replay attacks , the encryption has to be "randomized" : two encryptions of the same plaintext should not result in the same ciphertext. \\
The initial vector is used as first input of block cipher to ensure the uniqueness of the ciphetext, independently from the plaintext message : that's why it has to change with every encryption, and it is sent in plaintext along the ciphertext message. It can be implemented as a counter, but also with  a nonce.\\
The counter solution has the advantage to be simple : every time an encryption is done, just increment the variable. It has also the advantage of being asynchronous : since the IV's evolution is known, it's not useful to send the IV with the ciphertext, the sender incremented each time it encrypt a message, and the receiver each time it decipher it. However, you have to be careful to have a sufficient space for IV in order to mitigate against replay attacks (see the \cite{WEP} cipher use such IV mode and is vulnerable because IV's size is not sufficiently large )  . \\
The other way to randomize the encryption is to use a nonce : a arbitrary number which does not repeat twice the same value. The nonce is often created from the output of a random number generator (such as /dev/random or /dev/urandom ) with a sufficient pool of entropy to ensure a low probability of generating twice the same number. The nonce is then send along with the encrypted message. 

\subsection{ Cipher Block modes of operation }

Most of the times, the length of the plaintext message will be larger than the block size the cipher operates on : the message has to be split in several blocks. The encryption of those block has to think through, since it can jeopardize the whole system. \\
In this section will be presented the different ways to encode a plaintext using block cipher, using slight variations on the IV's introduction and each block cipher input, and describing their strength and weakness. There is one broken way to operate a block cipher ( ECB mode ) , two majors secure ones ( CBC and CTR ) , and other minors modes (GCM, ...). \\

\subsubsection{ Electronic Codebook Mode (ECM) } 

The most straightforward way to encrypt several blocks (and the most fragile one) : map the input block to the encryption function and concatenate the output to get the ciphertext message.

\subsubsection{ Cipher-Block Chaining (CBC) }

\subsubsection{ Counter (CTR) }

\subsubsection{ Cipher Feedback (CFM) }

\subsubsection{ Output Feedback (OFM) }


\subsection{Data Encryption Standard (DES)}

The DES is a US standard created by the US. DOD \footnote{Department of Defence} in the 70's and built upon Feistel Networks as PRF. It has been broken by "exhaustive search" attacks in 1997 (and by others ways since).

\subsubsection{Feistel Network}

The backbone architecture behind DES is a 16-round Feistel cipher. The Feisel cipher has interesting properties and it's construct in the following manner : the input message (plaintext or ciphertext) is split in two parts $L_0$ and $R_0$ and feed through the structure. At each "round", an arbitrary function $f_i$ scramble the input message $L_i$ before xor'ing it with $R_i$.

\begin{figure}[ht!]
    \centering
       \includegraphics[width=\textwidth]{images/Feistel_cipher_encryption.pdf}
    \caption{Feistel cipher encryption and decryption \\ ( the arbitrary functions are generated using the secret key and a key-schedule. )\\ source : Wikipedia}
	\label{fig:Feistel_cipher}
\end{figure}
 

\begin{mytheorem}[Mathematical Formula]
    \begin{align}
        \forall i \in \llbracket1,m\rrbracket,&                 \\
        &L_i = R_{i-1}                                          \\
        &R_i = f_i(R_{i-1}) \xor L{i-1}                         
    \end{align}
\end{mytheorem}

The system is invertible, even if the arbitrary functions $f_i$ aren't : to decrypt a cyphertext, just invert the order the function to be applied. 
\begin{align}   
    \forall i \in \llbracket1,m\rrbracket,&                     \\
    &R_i = L_{i+1}                                              \\
    & L_i = f_{i+1}(L_{i+1}) \xor R{i+1}                    
\end{align}

This property is really handy since it halves the cipher's implementation in terms of space, electrical consumption and parts prices.

\subsubsection{ DES construction }

The DES cipher works on 64-bit blocks encrypted by a 56-bit key, and it's constructed in the following manner : it has at first an initial permutation ($IP$), then a 16-round Feistel network with a key expansion algorithm for the arbitrary functions, and finally the inverted initial permutation $IP^{-1}$ applied.

\subsubsection{Festeil functions}

The DES use a specific algorithm to expand the secret key into a family a sub-key which will feed the Feistel family of PRF $f_0,f_1,\dots f_n$ : it is called a key schedule. Before describing the key schedule, let's see the Feistel function f. 

Each PRF $f_i$ use a 48-bit sub-key $k_i$ generated from the key-schedule and the 32-bit half-message from the previous round as input. The message is expanded to 48 bits to match the sub-key's length, and the former is xor'ed with the latter.\\
The result is split in 6-bit value arrays (8 values if you can count). Each one of these 6-bit arrays goes through a 6-to-4 bits substitution functions $S_i$. For speed purposes, those $S_i$ boxes are often implemented as lookup-tables. \\
The result from the boxes is then recollected in a 32-bit message which goes through a final constant permutation function P in order to obtain the output. \\

The 56-bit key is expanded into sixteen 48-bit sub-key using rotations and permutation ( see \cite{DES-wikipedia} ).

\begin{figure}[ht!]
    \centering
       \includegraphics[width=0.7\textwidth]{images/DES-f-function.png}
    \caption{DES Festeil Function \\ source : Wikipedia}
	\label{fig:DES-f-function}
\end{figure}


\begin{mytheorem}[Luby-Rackoff (1985)]
    Let $f:K\times\llbracket0,1\rrbracket^n -> \llbracket0,1\rrbracket^n$ a secure PRF, then the 3-round Feistel network is a secure PRP. \footnote{ you can see the proof here : \url{http://www.csc.kth.se/utbildning/kth/kurser/DD2448/krypto10/handouts/LR.pdf }}
\end{mytheorem}

The DES is a 16-round Feistel network, therefore it has the semantic security built on.  Even if the key schedule, S-boxes and initial permutation are known (they are fixed by the standard) it is not considered as a breach of security as long as the key remains unknown. The attacker may know the in and outs of the system, it should not be able to recover the plaintext message or the key.

\subsubsection{ Vulnerabilities }

The DES construction may be secure against cyphertext attacks in theory, it is not the case in reality. The most famous security hole is DES vulnerability against Exhaustive Search Attacks : 	

\subsubsection{Variants}

It exists some variants of the DES cipher, destined to lessen its flaw against brute force attacks. The two most known are 3DES (or "Triple DES" ) and DESx. \\
The 3DES is really straightforward : it's 3 DES ciphers in cascade. It has been designed to prevent exhaustive search attacks by expanding the key space necessary to compute. Since it use the same DES block, it is a fairly inexpensive (in terms of dev times or parts) upgrade from DES which bring a good layer of security.
The DESx cope up with brute force by using the key whitening technique : on top of the 56-bit key used for the DES cipher, the DESx uses also two 64-bit keys $K_1$ and $K_2$. The key $K_1$ is xor'ed with the plaintext before going through the DES cipher, while the key $K_2$ is xor'ed with the DES output to give the ciphertext. This more complex design makes the DESx more difficult tot crack by brute force attacks, as well as linear and differential attacks.

\subsection{Advanced Encryption Standard (AES)}

\subsubsection{Definition}

The AES cipher is the next generation of standard encryption, replacing DES. It has been created around 2000, and it currently widely use for symmetric encryption (disk encryption with TrueCrypt, Wifi with WPA, ...). AES is currently important since it's used for TLS transmissions, the backbone layer for HTTPS communications : in other words the present secure Web \footnote{while it starts to be recommended to use Elliptic Curve for TLS}.

\subsubsection{ AES cipher implementation }


\subsection{Cipher Block Chain construction (CBC)}



\begin{mytheorem}[CBC Semantic Security]
   $\forall L>0$, if E is a secure PRP over (K,X), \\
   then $E_{CBC}$ is a semantically secure cipher under CPA over $(K,X^L,X^{L+1})$. \\
 	Moreover : $Adv_{CPA}[A,E_{CBC}] \leq 2.Adv_{PRP}[B,E] + 2.\frac{q^2.L^2}{|X|} $. \footnote{where q is the number of messages encrypted using the key k, and L the length of the messages.}
\end{mytheorem}

\subsubsection{Construction with a Nonce-based IV}

\subsection{Randomized Counter Mode construction (CTR)}


\subsubsection{Theorem}

$\forall L>0$, if F is a secure PRF over (K,X), then $E_{CTR}$ is a semantically secure cipher under CPA over $(K,X^L,X^{L+1})$.

$Adv_{CPA}[A,E_{CTR}] \leq 2.Adv_{PRP}[B,F] + 2.q^2.L/|H| $
q : nb of messages encrypted using key k
H : length of the message

CTR construction is a bit better than CBC since the advantage is linear in the message's length (for CBC it is squared) and the algorithm is parallel (whereas it's sequential for CBC) which is easier to speed up (using FPGA or ASIC).






\chapter{Integrity}


This chapter will present the other aspect of cryptography : tampering prevention. An adversary can be able to do more than just eavesdropping : he can actually modify ciphertext messages on-the-go (using fro example a Man in the Middle attack). Therefore we need methods to ensure the non modification of the message during transmission.\\
In Dan Boneh's lecture, the integrity enforcement mechanisms are a minor part of the course, so this chapter will be quite succinct (especially since some generic constructions were already explained in the confidentiality chapter).


\section{Message Authentication Code (MAC)}

The most common way to enforce message integrity is to add a tag to the message, which will be verified upon reception. Moreover, the tag needs to be created using a secret key in order to prevent an attacker from fooling the verification algorithm.

\begin{mydef} $MAC = (S,V)$  
\begin{flushright}
\begin{minipage}[t]{0.8\textwidth}
\indent   	S : Tag Generator \\
\indent     $S: (k,m) -> t$   \\
\indent	V : Verification Algorithm \\
\indent    $V(k,m,t) -> {0,1}$ \\
\end{minipage}
\end{flushright}
\end{mydef}

The tag and tag generation are more often called respectively "hash" and "hashing"  : a hash function is an algorithm which takes variable-length data as input and outputs a fixed-length image of the input data. While being created in order to lessen the memory footprint of databases and speed-up the lookup of elements (hash tables, caches), hashing functions are also vastly used in cryptography. 
 
\subsection{Secure Mac}

The experiment needed to describe the security of a MAC mechanism is the same as for the cipher's semantic security : the attacker can submit 2 messages q times, and receive the tag of n ones. If it can't forge a new valid pair (m,t) to submit to V with a significant advantage, the MAC algorithm is considered secure.

\subsection{Collision Resistance}

The strength of a MAC against forged tags are closely related to the algorithm's resistance against collision attacks. A collision is a pair of messages $(m_0,m_1)$ such that $H(m_0) == H(m_1)$. We can clearly see that $m_0$ and $m_1$, if the MAC is using the hash functions $H$, have great chances to have the same tag. The verification algorithm will take one for another, which is a breach of security. \\
Therefore the algorithm for the tag generator has to be built upon collision resistant hashing functions.


\subsection{MAC Padding}

As for block ciphers, a hash algorithm works usually on a fixed length of plaintext information. However, contrary to the former, it is not possible to just pad the input text with 0's because it is insecure : the attacker can then send parts of the same message to retrieve important parts of information (block length, last digit digest ). \\
The standard (ISO) currently used is to pad with one '1' and the rest with '0' 's.


\section{Merkle-Damgard Paradigm}

\subsection{Compression functions}



\subsection{Constructions}

\subsubsection{Construction from PRF}

A secure MAC can be easily constructed from a PRF family. The following theorem is important since a lot of real-world MAC use it, in various environment (Internet, Banks, Defence, ..).

\begin{mytheorem}
    IF $F:K\times X \leftarrow Y$ is a secure PRF and $card(Y)$ is large, then $I_F = (F, V_F)$ is a secure MAC and : \\
    $ Adv_{MAC} \leq Adv_{PRF} + \frac{1}{|Y|} $
\end{mytheorem}

\subsubsection{CBC-MAC}
\subsubsection{HMAC}
\subsubsection{NMAC}
\subsubsection{PMAC}
\subsubsection{Carter-Wegman MAC}


\section{Authenticated Encryption}	


\chapter{Key Exchange Protocols}

This chapter will focus on the initialisation aspect of cryptography, and more exactly on the part which contains the key-exchange protocols. As we seen in the last two chapters, there are robust systems to ensure confidentiality and integrity of data communication against eavesdropping and active attackers (given  those systems were correctly implemented). However, these mechanisms does not describe how the two actors (generally called Alice and Bob) create a shared secret (i.e. key) used to encrypt data. \\
Key-exchange protocol is the last part of Coursera Cryptography I class and it's more detailed in the second class.

\section{Trusted 3rd party}

The most straightforward method for key exchange between two parties (or more) is to use a third one as proxy/notary. Each party gives their secret key to the TTP\footnote{Trusted Third Party} and, in return gives a shared one. While having many implementations (notary for wills and contract, Paypal for money transactions, certifications authorities (CA) ...) this procedure is not completely secure since it create a single point of failure (e.g. the TTP being pwn'ed) and the ``trusted'' part is sometimes false (as seen with the NSA leak ) : that's why it is preferable to look at zero-knowledge protocols for key exchange.


\section{Merkle Puzzles}
Merkle puzzles has been developed as a way to exchange keys using a generic symmetric cipher between two persons without a third-party. It's constructed upon a ``puzzle'', which means a computational difficult problem.\footnote{I rather not try to define what a ``difficult'' computational problem means, let's say it's more or less NP-hard problems.} \\
As usual , we are in a situation of security against eavesdropping ( Merkle puzzles are insecure against active attacks such as AP poisoning ). Eve wants to retrieve the shared key that will be use for data transmission, and can only listen the communications between Alice and Bob. \\

Steps for key-exchange using Merkle puzzles :
\begin{enumerate}
	\item Alice prepares n problems, and send it to Bob. Each problem has a message encrypted containing an identifier and a secret key
	\item Bob solves one, and send the identifier in plaintext back to Alice
	\item Alice fetch the secret key corresponding to the identifier sent by Bob
	\item the two parties can communicate securely using their shared key.
\end{enumerate}

\paragraph{Complexity}

The strength of the Merkle puzzle scheme lies in the asymmetry of the problem. Let a Merkle Puzzle be of complexity linear $O(m)$.  Alice send n puzzles to Bob which solves one so Bob need $O(n+m)$ time computation. Bob sending back an identifier, Alice only need to solve one puzzle, thus $O(n+m)$ too. However, Eve needs to solve ``all'' (until the identifier is found) puzzles, so $O(n*m)$ : Alice and Bob needs linear time computation, while Eve needs quadratic time.\\

While being quite useful, the best gap in complexity provided by Merkle Puzzles is quadratic at best : for many real-world cases it is not enough\footnote{Merkle Puzzles are also vulnerable to quantum computation}. That's why exponential time gap scheme has been created (by Diffie and Hellman or RSA). 

\section{Public Key Encryption}

The public key encryption class regroup a list of ciphers useful for key-exchange. They all rely on the same idea as Merkle Puzzle's, meaning finding a problem which is exponentially harder to solve than checking if a solution is a valid one.


\subsection{Diffie-Hellman protocol}
The Diffie-Hellman protocol is a key exchange scheme created in 1976, but still fairly used nowadays.  The strength of this protocol is based on the difficulty of the discrete logarithm problem. 

Steps : 
\begin{itemize}
	\item Alice and Bob choose a finite group (generally $\frac{\mathbb{Z}}{p\mathbb{Z}}$) and g a generator from this group. 
	\item Alice choose randomly a number $a$, and send to Bob $g^a$
	\item Bob does the same with $b$
	\item $g^{a.b}$ is the shared secret used to encrypt communications. Any eavesdropper has access to $g^{a}$ and $g^{b}$, but cannot easily compute the shared secret from these two numbers (since he has to compute either $a$ or $b$ using discrete log).
\end{itemize}

Discrete exponentiation is fairly easy ( linear time ) whereas discrete log is hard (best known algo in $exp(O(\sqrt[3]{x}))$ : this protocol use the asymmetry of the operation $\times $ to protect the key exchange, thus gaining the name of asymmetric-cryptography. However, if it were to be found an effective to the discrete log problem, the Diffie-Hellman would become insecure (as many others systems like El-Gamal).\\

This exchange is secure against eavesdropping, but not against Man-in-the-Middle attacks. The mitigation against it is to send, as well as $g^x$, a publicly certified signature from a trusted third party\footnote{More on digital signature in the second part of the Coursera course}.\\

\paragraph{Multi-party communication}
As seen previously the Diffie-Hellman protocol gives a secure way for two parties to exchange a secret. What about more than two people ? Unfortunately, this a currently an open problem \footnote{there is nonetheless a simple solution for 3 parties.} : there is no efficient way to create a multi-party shared secret.



% 3 tools : 
% Asymmetric encryption for key-exchange : ( G, F, F-1 )
% Symmetric encryption for communication : (E, D)
% Hash function for integrity			 :  H

% Encryption :
% G() -> (pk, sk)

% x is chosen randomly
% E(pk, m) ->  |F(pk,x)||Es( H(x), m )|

\subsection{RSA encryption}


\paragraph{Trapdoor Functions}

RSA encryption relies heavily on trapdoor functions for security. Trapdoor functions - or ``one-way functions'' - are easily computable in one direction, yet difficult in the other one \footnote{``easy'' and ``hard'' does not have any formal meaning, we just look at difficulty from a empirical point-of-view}. \\

In cryptography, a trapdoor function is a triplet $(G, F, F_{-1})$ where :
\begin{itemize}
	\item $G$: $ seed \rightarrow (pk, sk) $ a randomized algorithm which produce a public key and a secret one.
	\item $F$ : $ (pk, X) \rightarrow  Y $ used to encrypt message $X$ using public key $pk$.
	\item $F_{-1}$ : $ (sk, Y) \rightarrow X $ used to decrypt ciphertext $Y$ using secret key $sk$.
\end{itemize}

\begin{mytheorem}
	$\forall (pk,sk) $ generated by $G$, \\
	$\forall m \in M $, $F_{-1}( sk, F(pk, m) ) = m $
\end{mytheorem}

\paragraph{RSA Trapdoor}

The RSA trapdoor function relies on the prime factorization problem and the modular e-roots' problem. \\
First, the prime factorization problem : given two prime numbers, it's easy to compute their product. However, given the product of two primes numbers, it is fairly difficult to find the factorization. That's mean we can communicate the primes' product publicly without lessening the security of the protocol - the two primes still has to be big enough -. In the case of RSA, the primes' product number will be used as a generator for (Z/nZ).\\
Secondly, the modular e-roots' problem : given a number and an exponent, it is easy to compute its exponentiation in Z/nZ, but it is difficult to compute the e-root in the same group (in the math sense).

\paragraph{ECC Trapdoor}

Another famous trapdoor is the Elliptic Curve (Cryptography) trapdoor, which is more secure than RSA's trapdoor for the same key-length. \\

\begin{mydef}[Elliptic Curve]
An elliptic curve has the following equation in the plane : $P[x] = Q[y]$ where P and Q are polynoms of degree 3.
\end{mydef}

The Elliptic Curve trapdoor is a bit more complicated than modular e-roots and need some solid geometric bases. You can find a primer here : \url{http://blog.cloudflare.com/a-relatively-easy-to-understand-primer-on-elliptic-curve-cryptography}


\subsection{El Gamal}

El Gamal use Diffie-Hellman protocol to create a secure public-key protocol. It's currently used by GPG, some version of PGP and the algorithm isn't patented like RSA's. 

\paragraph{Definition \\}

ElGamal is built from a cyclic finite group $G$ ( $\mathbb{Z}_p^*$ for example, with p a ``big'' prime number ) : let $g$ be a generator\footnote{a generator g can build the set G only from it's geometric sequence : ${g^i, i\in N} == G$.}  of $G$. Like any other authenticated encryption system, it has a symmetric cipher $(E,D)$ for encryption and a hash function $H$ for integrity.

\paragraph{Key Generation :}
 Alice and Bob choose respectively a and b from the set of powers used to generate $G$, and send to the other party resp. $g^a$ and $g^b$ which are used in the Diffie-Hellman protocol. 
 
\begin{itemize}
	\item Alice's secret key : $a$
	\item Alice's public key : $pk_A = (g, g^a)$
	\item Bob's secret key : $b$
	\item Bob's public key : $pk_B = (g, g^b)$
\end{itemize}
 
\paragraph{Encryption :}

Alice wants to send a message $m$ to Bob, using Bob's public key. First, Alice create the encryption key $k$ using the hash function $H$ : $k = H(g^b, g^{a.b}) $. Then, using $k$, Alice encrypt the message : $c = E(k,m)$. Alice output $(g^a, c)$

\paragraph{Decryption :} 
The decryption is really straightforward : Bob, using his secret key $b$ can compute $g^{a.b}$ from $g^a$ sent by Alice along with the message. $g^{a.b}$ is then used by the hash function to compute the secret key $k$ which will be used by the decipher $D$.

\paragraph{Security : \\}

The ElGamal security theorem is built upon Diffie-Hellman security, and especially Diffie-Hellman Computational/Decisional Assumption\footnote{The Decisional version is a generalisation of the Computational version. For more infos : \url{http://en.wikipedia.org/wiki/Decisional_Diffie\%E2\%80\%93Hellman_assumption}. )}.

\begin{mytheorem}[DH Computational Assumption]
Let $G$ a cyclic finite group, and $g$ a generator of $G$.
Then, $\forall$ efficient algorithms $A$,
\begin{flushright}
	$Pr[ A(g,g^a,g^{b}) == g^{a.b} ]$ is negligible.
\end{flushright} 
\end{mytheorem}
 
 In layman terms, given the generator and the partial public keys (that's what the attacker has access to), it is almost impossible to find an algorithm which can compute $g^{a.b}$.


\chapter{Miscellaneous}

In this chapter, we will cover various minor parts of cryptography which does not belong to the previous chapters, e.g. stenography, key derivation, deterministic encryption, ...

\section{Stenography}
% "Hide in plain sight"

\section{Key derivation}

\section{Deterministic Encryption}


%%%%%%%%%%%%%%
%% Summary

%%%%%%%%%%%%%%
%% Cites


\end{document}
%%%%%%%%%%%%%%%%%%%%%%%%%%%%%%%%%%%%%%%%%%%%%%%%%%%%%%%%%%%%%%%%
%% 
%%%%%%%%%%%%%%%%%%%%%%%%%%%%%%%%%%%%%%%%%%%%%%%%%%%%%%%%%%%%%%%%